%%%%%%%%%%%%%%%%%%%%%%%%%%%%%%%%%%%%%%%%%%%%%%%%%%%%%%%%%%%%%%%%%%%%%%%%%%%%%
% Chapter 6: Summary and Conlusions
%%%%%%%%%%%%%%%%%%%%%%%%%%%%%%%%%%%%%%%%%%%%%%%%%%%%%%%%%%%%%%%%%%%%%%%%%%%%%%%

%++++++++++++++++++++++++++++++++++++++++++++++++++++++++++++++++++++++++++++++
\subsection{Conclusions}

Through the development of this thesis we have reach the following conclusions: 

First of all, the fact of work with algorithms which use high amount of parameters, like CMA-ES or OBLCPSO, increase the complexity of evaluate the performance of the algorithms due to the fact that the number of possible tests increase with each parameter of the algorithm. Even though, the self-regulation nature of the algorithms like CMA-ES eases this task.

On the other hand, the structure of the developed algorithms are truly adaptable since we could add, in all of them, the different diversify techniques proposed in sections \ref{sec:OBL} y \ref{sec:BG}.

Furthermore, as a result of the high amount of proposed functions by the contest GenOpt, the task of evaluate new modifications was really dificult due to the fact that many times, the results only were improved in one function and, altogether, the results stayed almost inmutable. \\

Finally, taking into account the obtained results by the developed algorithms, we can conclude that the algorithm CMA-ES is the most effective solving the proposed functions by the contest GenOpt.



%%%%%%%%%%%%%%%%%%%%%%%%%%%%%%%%%%%%%%%%%%%%%%%%%%%%%%%%%%%%%%%%%%%%%%%%%%%%%%%
\subsection{Future work}

The developed work in this project could serve up to test the performance of different meta-heuristic algorithms for solving continuos optimization problems.\\

Moreover, the future work of this project is improving the developed algorithms, as well as deeply research the parameters of the algorithms, in the interest of participate in the next edition of the contest GenOpt.  
