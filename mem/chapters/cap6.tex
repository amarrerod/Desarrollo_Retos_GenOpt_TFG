%%%%%%%%%%%%%%%%%%%%%%%%%%%%%%%%%%%%%%%%%%%%%%%%%%%%%%%%%%%%%%%%%%%%%%%%%%%%%
% Chapter 6: Summary and Conlusions
%%%%%%%%%%%%%%%%%%%%%%%%%%%%%%%%%%%%%%%%%%%%%%%%%%%%%%%%%%%%%%%%%%%%%%%%%%%%%%%

%++++++++++++++++++++++++++++++++++++++++++++++++++++++++++++++++++++++++++++++
\section{Conclusions}

Through the development of this degree thesis we have reached the following conclusions: 

First of all, the fact of working with algorithms which use a high amount of parameters, like CMA-ES or OBL-CPSO, increases the complexity to evaluate the performance of the algorithms due to the fact that the number of possible tests increases with each parameter. Even though, the self-regulation nature of the algorithms like CMA-ES facilitates this task.

Furthermore, as a result of the large amount of proposed functions by the GenOpt contest, the task of assessing a new modification was really difficult. In many times, results were only improved for a low number of the proposed functions. \\

It is worth to be mentioned that the CMA-ES algorithm (Section \ref{sec:CMA}) accomplished the \textbf{third place} in the final leaderboard of the GenOpt contest considering the High Jump criterion.

Finally, taking into account the obtained results by the developed algorithms, we can conclude that the algorithm CMA-ES was the most effective approach solving the functions proposed by the GenOpt contest.



%%%%%%%%%%%%%%%%%%%%%%%%%%%%%%%%%%%%%%%%%%%%%%%%%%%%%%%%%%%%%%%%%%%%%%%%%%%%%%%
\section{Future work}

The work developed in this project could be a starting point to test the performance of different meta-heuristic algorithms for solving continuos optimization problems.\\

A potential line of future work may be the improvement of the different tested algorithms, with the aim of participating in future editions of the GenOpt contest.
