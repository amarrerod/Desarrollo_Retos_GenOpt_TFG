%%%%%%%%%%%%%%%%%%%%%%%%%%%%%%%%%%%%%%%%%%%%%%%%%%%%%%%%%%%%%%%%%%%%%%%%%%%%%
% Chapter 5: Conclusiones y Trabajos Futuros 
%%%%%%%%%%%%%%%%%%%%%%%%%%%%%%%%%%%%%%%%%%%%%%%%%%%%%%%%%%%%%%%%%%%%%%%%%%%%%%%

%%%%%%%%%%%%%%%%%%%%%%%%%%%%%%%%%%%%%%%%%%%%%%%%%%%%%%%%%%%%%%%%%%%%%%%%%%%%%%%
\section{Conclusiones}

Durante la elaboración de este Trabajo de Fin de Grado hemos obtenido varias conclusiones que destacaremos a continuación.

En primer lugar, trabajar con algoritmos que requieren una gran cantidad de parámetros, como es el caso del algoritmo CMA-ES u OBL-CPSO, hace que la evaluación del rendimiento del algoritmo sea más compleja dado que por cada parámetro se incrementa significativamente la cantidad de experimentos a realizar para probar el rendimiento del mismo. Aún así, en el caso del algoritmo CMA-ES, la naturaleza auto-reguladora del algoritmo facilita en cierta medida esta tarea. 

Hemos de destacar además que, debido al gran número de funciones propuestas por el concurso GenOpt, las modificaciones realizadas a los algoritmos resultaron difíciles de evaluar dado que en bastantes ocasiones, la mejora era simplemente en funciones concretas y en conjunto, los resultados no variaban significativamente para considerar dicha modificación.\\

Merece la pena mencionar que el algoritmo CMA-ES (Sección \ref{sec:CMA}) consiguió el \textbf{tercer mejor puesto} según el criterio \textbf{High Jump} en la clasificación final del concurso GenOpt.

Finalmente, y considerando los resultados obtenidos por las técnicas implementadas, podemos concluir que el algoritmo CMA-ES es el más eficaz resolviendo el conjunto de problemas propuesto por GenOpt.


%%%%%%%%%%%%%%%%%%%%%%%%%%%%%%%%%%%%%%%%%%%%%%%%%%%%%%%%%%%%%%%%%%%%%%%%%%%%%%%
\section{Líneas de Trabajo Futuras}

El trabajo desarrollado en este proyecto puede servir como base para comprobar el rendimiento de diferentes técnicas meta-heurísticas a la hora de resolver problemas de optimización continua. \\
Además, como líneas de trabajo futuras se prevé la mejora algunas de las técnicas implementadas, así como un estudio más intensivo de los parámetros, con vistas a participar en la siguiente edición del concurso GenOpt.