
\begin{algorithm}[!ht]
  \caption{Búsqueda global(\mbox{})}
  \label{pseu:bg}
  \begin{algorithmic}[1]
    \STATE NumIndividuos = $\left | S \right |$;
    \STATE OrdenarPoblacion(S);
    \STATE MarcarNoExplorados(S);
    \STATE Centroide = CalcularCentroide();
    \STATE NumeroMejora = 0;
    \STATE NumeroExplorado = 0;
    \WHILE{$NumeroMejora > 0 $ y $NumeroExplorado < \left | S \right |$}
      \STATE k = 0;
      \WHILE{$S[k] = explorado $ y $NumeroExplorado < \left | S \right |$}
        \STATE k = rand(0, $\left | S \right |$); Buscamos un individuo $k$ no explorado
      \ENDWHILE
      \STATE S[k] = explorado;
      \STATE NumeroExplorado = NumeroExplorado + 1;
      \STATE Mejora = true;
      \WHILE{$Mejora = true$}
        \WHILE{$|a_{1}| + |a_{2}| + |a_{3}| \neq 1$}
          \STATE GenerarRand(a1, a2, a3); Obtenemos tres valores generados aleatoriamente de manera uniforme en el rango [0, 1] para realizar la modificación del individuo
        \ENDWHILE
        \WHILE{$ r_{1} < k$}
          \STATE $ r_{1}$ = rand(0, $\left | S \right |$); Buscamos un individuo $r_{1}$ aleatoriamente para usar sus variables en la generación del nuevo individuo
        \ENDWHILE
        \STATE NuevoInd = ModificarIndividuo(k, a1, a2, a3, Centroide, $r_{1}$);
        \IF{$NuevoInd < S[k]$}
          \STATE $Mejora = true;$
          \STATE $S = S \cap NuevoInd$;
          \STATE NumeroMejora = NumeroMejora + 1;
          \ELSE
            \STATE Mejora = false;
        \ENDIF
      \ENDWHILE
    \ENDWHILE
    \STATE OrdenarPoblacion(S); 
    \STATE S = ObtenerMejores(0, NumIndividuos, S);
    \RETURN $\left | S \right |$ mejores individuos encontrados
  \end{algorithmic}
\end{algorithm}
